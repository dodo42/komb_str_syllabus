\chapter{Blokové plány}

\section{Definícia, základné vlastnosti}

\begin{definition}

Vyvážený nekompletný blokový plán (angl. \emph{balanced incomplete block design}) $BIBD(v, b, r, k, \lambda)$ 
je usporiadaná dvojica $(X, \mathcal{B})$, kde $X$ je množina objektov a $\mathcal{B} \subset \powerset(X)$ je 
množina podmnožín objektov (tieto podmnožiny voláme \emph{bloky}), pričom sú splnené nasledujúce podmienky:

\begin{enumerate}
    \item $v = |X|$ je mohutnosť množiny objektov.
    \item $b = |\mathcal{B}|$ je mohutnosť množiny blokov.
    \item každý blok má mohutnosť $k$.
    \item každý bod je obsiahnutý v práve $r$ blokoch.
    \item každá dvojica bodov sa vyskytuje v práve $\lambda$ blokoch. 
\end{enumerate}
\end{definition}

\begin{theorem}
$\exists BIBD(v, b, r, k, \lambda) \Longleftrightarrow $ $\lambda$-násobný kompletný multigraf rádu~$v$ $\lambda K_v$
sa dá rozložiť na $b$ hranovo disjunktných klík rádu~$k$ ($K_k$).
\end{theorem}

\begin{theorem}

Nech existuje $BIBD(v, b, r, k, \lambda)$. Potom:
\begin{enumerate}
    \item $vr = bk$
    \item $\lambda (v-1) = r (k-1)$
\end{enumerate}
\end{theorem}

\begin{corollary}
Preto namiesto značenia $BIBD(v, b, r, k, \lambda)$ budeme často
použivať značenie $BIBD(v, k, \lambda)$, nakoľko 
zvyšné parametre vieme dorátať: 
$$r := \dfrac{\lambda (v-1)}{k-1},~ b := \dfrac{\lambda v (v-1)}{k (k-1)}$$
\end{corollary}

\begin{theorem}

Nech existuje $BIBD(v, b,r, k, \lambda)$, kde $X = \set{x_1, x_2, \ldots, x_v}$ a $\mathcal{B} = \set{B_1, \ldots, B_b}$. 
Nech matica incidencie $A \in \set{0, 1}^{v \times b}$ je matica typu $v\times b$, kde $A_{ij} = 1$ práve vtedy, keď $x_i \in B_j$.
Potom $A A^T = (r-\lambda) I_v + \lambda J_{v}$, 
kde $I_v$ je matica identity rádu $v$ a 
$J_v$ je matica jednotiek typu $v \times v$.
\end{theorem}

\begin{lemma}
Nech $A$ je matica incidencie blokového plánu $BIBD(v, b,r, k, \lambda)$. Potom $det(AA^T) = (r-\lambda)^{v-1} (v\lambda - \lambda + r)$.
\end{lemma}

\begin{corollary}
Ak $BIBD(v, b,r, k, \lambda)$ je blokový plán a $b=v$, tak matica incidencie $A$ je regulárna a matici $A^T$ tiež zodpovedá nejaký blokový plán.
\end{corollary}

\begin{remark}
Blokové plány také, že $b = v$, voláme symetrické.
\end{remark}


\begin{theorem}{(Fisherova nerovnosť)}
Nech existuje blokový plán $BIBD(v, b,r, k, \lambda)$. Potom $b \geq v$.
\end{theorem}

\section{Cyklické blokové plány a diferenčné množiny}

\begin{definition}

Množina $D = \set{d_1, \ldots, d_k} \subset \mathbb{Z}_v$ mohutnosti $k$ sa volá $(v, k, \lambda)$-diferenčnou množinou, ak 
pre každý nenulový prvok $a \in \mathbb{Z}_v$ existuje práve $\lambda$ usporiadaných dvojíc $(d_i, d_j) \in D^2$ takých, že
$d_i - d_j \equiv a (mod~v)$. 
\end{definition}

\begin{remark}

Množina $\set{0, 1, 3}$ je $(7, 3, 1)$-diferenčnou množinou.

\end{remark}

\begin{remark}

Podobným spôsobom je možné definovať diferenčné množiny nad konečnými grupami rádu $v$.

\end{remark}

\begin{definition}
$(v, k, \lambda)$-BIBD je cyklický, ak existuje permutácia s cyklom dlžky $v$ taká, že zachováva bloky\footnote{takéto
zobrazenia sa všeobecne nazývajú \emph{automorfizmy}}. 
Formálne, blokový plán je cyklický, ak 
existuje permutácia  $\phi \in S_v$ s cyklom dlžky $v$ taká, že 
$$\mathcal{B} = \set{\set{\phi(x_1), \ldots, \phi(x_k)} ~|~ \set{x_1, \ldots, x_k} \in \mathcal{B} }$$
\end{definition}

\begin{theorem}
Množina $D = \set{d_1, \ldots, d_k}$ je $(v, k, \lambda)$-diferenčná množina práve vtedy, keď
$(X, \mathcal{B})$, kde $X = \mathbb{Z}_v$ a $\mathcal{B} = \set{D + i ~|~ \forall i \in \mathbb{Z}_v}$ ($D + i := \set{d_1 + i, \ldots, d_k + i}$)
je cyklický $(v, k, \lambda)$-BIBD. 
\end{theorem}

\begin{definition}
Nech $F$ je konečné pole. Nech $V \cong F^{n+1}$ je vektorový priestor dimenzie $n+1$ nad poľom $F$. 
Definujeme reláciu $\sim$ nad prvkami $V^* := V - \set{\vec{0}}$:

$$\forall \vec{a},\vec{b} \in V^*: \left( \vec{a} \sim \vec{b} \overset{def}{\Longleftrightarrow} \exists k \in F: \vec{a} = k \vec{b} \right)$$

Potom rozklad $V^*$ na triedy ekvivalencie $\mathbb{P}^n(V) := \faktor{V^*}{\sim}$ je $n$-rozmerná projektívna rovina nad $F$.

Projektívnu rovinu dimenzie $n$ nad konečným poľom s $q = p^r$ prvkami oznáčujeme ako $PG(n, q) := \mathbb{P}^n\left( (\mathbb{Z}_p)^r \right)$

\end{definition}

\begin{theorem}{(Typ S dif. množín --- Singerove dif. množiny)\footnote{\TODO je to bez dokazu ci s dokazom?}}\\
Nech množina $D$ obsahuje všetky nadroviny konečnej projektívnej roviny $PG(n, q)$ 
(nadrovina je faktorový obraz vektorového podriestoru dimenzie $n$). 
Potom $D$ je $(v, k, \lambda)$-diferenčná množina s parametrami:
$$v = \dfrac{q^{n+1}-1}{q-1}, k = \dfrac{q^n - 1}{q-1}, \lambda = \dfrac{q^{n-1}-1}{q-1}$$
\end{theorem}

\begin{theorem}{(Typ Q dif. množín --- kvadratické reziduá)}
\end{theorem}

\begin{theorem}{(Typ B dif. množín --- bikvadratické reziduá)}
\end{theorem}


\section{Hadamardove matice}
\section{Konečné projektívne roviny}
\section{Steinerovské systémy trojíc, zovšeobecnenia}
